\documentclass[11pt]{ltjsarticle}

\usepackage[haranoaji, nfssonly]{luatexja-preset}
\usepackage{graphicx}
\usepackage{url}

\newcommand{\tabref}[1]{表~\ref{#1}}
\newcommand{\figref}[1]{図~\ref{#1}}
\newcommand{\qeuref}[1]{式~\ref{#1}}


\title{UNO説明資料}
\author{菅原直輝}

\begin{document}

\thispagestyle{empty}
\vspace*{20pt}

\begin{flushright}
  2020年04月29日

  千葉工業大学 菅原直輝
\end{flushright}

\vspace*{100pt}
\begin{center}
{\Huge UNO説明資料}
\end{center}

\newpage

\pagenumbering{roman}
\tableofcontents

\clearpage
\pagenumbering{arabic}

\section{UNOとは}
UNOとは専用のカードを用いて遊ぶゲームである.配られた手札を早くなくした人が勝者となる.
プレイヤーの人数は2~10人.UNOの語源はスペイン語・イタリア語で「1」を意味する「uno」から由来する\cite{UNO-AMAZON}.


\section{遊び方}
\subsection{準備}
ゲームを開始する前に,カードを配るなどの準備が必要となる\cite{日本ウノ協会公認ルール}.
\begin{enumerate}
  \label{ゲーム準備}
  \item 全員が裏返しで1枚ずつカードを引き,一番数字が大きい人(記号カード,特殊カードは0とする)をスタートプレイヤーとする.
  もしくはじゃんけんによって決める.
  \item スタートプレイヤーはカードをよくシャッフルし,スタートプレイヤーの右隣の人がカットし,それを左から順番に1枚ずつ7枚を全員に配布する.
  \item スタートプレイヤーは山札の一番上のカードを場のカードとする.このとき,ワイルド・ドローフォーのときは山札をシャッフルしてやり直す.
  ドローツー,スキップ,リバースの際は,スタートプレイヤーがそれを出したように扱う.
\end{enumerate}
\subsection{ゲーム進行}
% TODO: 文章を入れる
\begin{enumerate}
  \item スタートプレイヤーの左隣のプレイヤーから時計周りに同じ色のカード,同じ数字のカードを出していく.
  \item 場にカードを出せない場合は,山札からカードを1枚引く.
  \item 山札から引いたカードをそのまま場に出すこともできる.
  \item 山札がなくなった場合,一番上をのぞく場のカードをスタートプレイヤーがシャッフルし,その後左隣のプレイヤーがカットしあらためて山札とする.
  \item 手札が残り1枚になったときは``ウノ''と宣言する.詳しくは\ref{sec:``ウノ''の宣言}に示す.
  \item 1人のプレイヤーの手札が0枚になった場合,1ラウンドの終了となり各プレイヤーのスコアを出す.スコアリングについては\ref{sec:スコアリング}に示す.
\end{enumerate}

\newpage

\section{スコアリング}
\label{seac:スコアリング}

\subsection{ポイント}

あがったプレイヤーは他のプレイヤー全員の手札の合計ポイントが得点となる.
また他のプレイヤーは自分の手札の合計ポイントだけ減点される.

各カードのポイントを\tabref{tab:カードポイント}に示す.

\begin{table}[h]
  \begin{center}
    \caption{各カードのポイント}
    \begin{tabular}{|l|c|} \hline
      カード & ポイント \\ \hline \hline
      各数字カード & カードに書いてある0~9ポイント \\
      各記号カード & 各20ポイント \\
      特殊カード & 各50ポイント \\ \hline
    \end{tabular}
    \label{tab:カードポイント}
  \end{center}
\end{table}

\subsection{順位}
全5ラウンドを通じて,最終的に得点が高い順から1位,2位…と順位を決定する\cite{日本ウノ協会公認ルール}.


\section{利用するカード}
UNOでは専用のカードが利用される.カードは大きく次の3種類に分類される.

\begin{itemize}
  \item 赤,青,黄,緑の色がついた0~9の数字カード.詳細は\ref{sec:数字カード}にて記述する.
  \item 赤,青,黄,緑の色がついた記号カード.詳細は\ref{sec:記号カード}にて記述する.
  \item 色のない特殊カード.詳細は\ref{sec:特殊カード}にて記述する.
\end{itemize}

\subsection{数字カード}
\label{sec:数字カード}
数字カードは赤,青,黄,緑の4色があり,0~9の数字で構成される.
0は各色2枚.1~9は各色2枚ずつで構成される.
数字カードは,場のカードと同じ数字のカードのみ出せる.
数字カードの例を\figref{fig:数字カードの例}に示す.

\begin{figure}[h]
  \begin{center}
    \includegraphics[width=0.25 \linewidth]{assets/UNO_数字カード.pdf}
    \caption{数字カードの例}
    \label{fig:数字カードの例}
  \end{center}
\end{figure}


\newpage

\subsection{記号カード}
\label{sec:記号カード}
記号カードは赤,青,緑,黄の色があり,場のカードと同じ色のカードのみ出せる.各色2枚ずつで構成される\cite{日本ウノ協会公認ルール}。

\subsubsection{ドローツー}
次の手番のプレイヤーは山札からカードを2枚引く.
このカードで山札からカードを引かされたプレイヤーの手番はそこで終了し,
場にカードを出すことはできない\cite{日本ウノ協会公認ルール}.
\figref{fig:ドローツーカードの例}にドローツーカードの例を示す.

\begin{figure}[h]
  \begin{center}
    \includegraphics[width=0.25 \linewidth]{assets/UNO_記号カード.pdf}
    \caption{ドローツーカードの例}
    \label{fig:ドローツーカードの例}
  \end{center}
\end{figure}

\subsubsection{リバース}
手番が反対回りになる.2人でプレイしている場合はスキップと同じ効果をもつ\cite{日本ウノ協会公認ルール}.
\figref{fig:リバースカードの例}にリバースカードの例を示す.

\begin{figure}[h]
  \begin{center}
    \includegraphics[width=0.25 \linewidth]{assets/UNO_リバースカード.pdf}
    \caption{リバースカードの例}
    \label{fig:リバースカードの例}
  \end{center}
\end{figure}

\subsubsection{スキップ}
次のプレイヤーの手番を飛ばす\cite{日本ウノ協会公認ルール}.
\figref{fig:スキップカードの例}にスキップカードの例を示す.


\begin{figure}[h]
  \begin{center}
    \includegraphics[width=0.25 \linewidth]{assets/UNO_スキップカード.pdf}
    \caption{スキップカードの例}
    \label{fig:スキップカードの例}
  \end{center}
\end{figure}

\newpage

\subsection{特殊カード}
\label{sec:特殊カード}
特殊カードは色がなく,場のカードによる制限を受けない.
特殊カードは各種4枚ずつで構成される.

\subsubsection{ワイルド}
場のカードに関わりなく自分の手番ならいつでも出せ,さらに場の色を指定できる\cite{日本ウノ協会公認ルール}.
\figref{fig:ワイルドカードの例}にワイルドカードの例を示す.

\begin{figure}[h]
  \begin{center}
    \includegraphics[width=0.25 \linewidth]{assets/UNO_ワイルドカード.pdf}
    \caption{ワイルドカードの例}
    \label{fig:ワイルドカードの例}
  \end{center}
\end{figure}

\subsubsection{ワイルド・ドローフォー}
場のカードに関わりなく,自分の手札に場の色と同じ色のカードを持っていなければ,いつでも出せる.
場の色を指定でき,次の手番のプレイヤーに山札から4枚引かせる.
カードを引いたプレイヤーの手番はそこで終了する.
なお,カードを引くこととなるプレイヤーには,
カードを引く代わりに,\ref{sec:チャレンジ}に示す「チャレンジ」をする権利がある\cite{日本ウノ協会公認ルール}.
\figref{fig:ワイルド・ドローフォーカードの例}にワイルド・ドローフォーカードの例を示す.

\begin{figure}[h]
  \begin{center}
    \includegraphics[width=0.25 \linewidth]{assets/UNO_特殊カード.pdf}
    \caption{ワイルド・ドローフォーカードの例}
    \label{fig:ワイルド・ドローフォーカードの例}
  \end{center}
\end{figure}

\newpage

\section{ルール}
\subsection{チャレンジ}
\label{sec:チャレンジ}
ワイルド・ドローフォーを出された次の手番のプレイヤーは
直前のプレイヤーが,本当に場のカードと同じ色のカードが手札にないか``チャレンジ''と宣言してチェックできる.
チャレンジされたプレイヤーはチャレンジしたプレイヤーにのみ手札を公開する.

同じ色のカードが手札にあった場合,
チャレンジ成功となりワイルド・ドローフォーを出したプレイヤーが4枚のカードを引かなければならない.
またワイルド・ドローフォーのカードは手札に戻し,正しいカード(場のカードと同じ色のカードやワイルドカードなど)を出す.

同じ色のカードが手札になかった場合,
チャレンジ失敗となりチャレンジしたプレイヤーがチャレンジのペナルティ分である2枚を含めて合計6枚のカードを引く.
次の色指定は通常どおりにワイルド・ドローフォーを出したプレイヤーが決める\cite{日本ウノ協会公認ルール}.

\subsection{``ウノ''の宣言}
\label{sec:``ウノ''の宣言}
手札が残り1枚になったときは``ウノ''と宣言する必要がある.
宣言を忘れた場合は,ペナルティとして山札からカードを2枚引く.
ただし,次のプレイヤーが場にカードを出した後の指摘ではペナルティは無効となり,
カードを引かなくてよい\cite{日本ウノ協会公認ルール}.

\newpage

\bibliographystyle{junsrt}
\nocite{*}
\bibliography{DB}
\end{document}